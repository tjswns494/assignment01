\documentclass[a4paper]{article}

\usepackage{fullpage} % Package to use full page
\usepackage{parskip} % Package to tweak paragraph skipping
\usepackage{tikz} % Package for drawing
\usepackage{amsmath}
\usepackage{hyperref}

\title{Assignment01}
\author{Sunjun Hwang}
\date{20151766}

\begin{document}

\maketitle

\section{Begin to use the source control website github}


I cloned the git hub to my local computer using the $<$git clone url address$>$ command.
After that, I added the file test.txt on the local computer that was cloned.
Now I have $<$git add test.txt$>$ command and $<$git commit -m "test add"$>$ to move to the file head from my local machine. Finally, the changes in the head are saved to the remote hub of the hub via the $<$git push origin master$>$ command.

\begin{figure}[!htbp]
\begin{center}
\includegraphics[scale=1.0]{git_clone1.PNG}
\end{center}
\caption{git clone to my local}\label{exampleplot}
\end{figure}
\begin{figure}[!htbp]
\begin{center}
\includegraphics[scale=0.7]{git_clone.PNG}
\end{center}
\caption{git clone to my local result}\label{exampleplot}
\end{figure}
\begin{figure}[!htbp]
\begin{center}
\includegraphics[scale=1.0]{add_test.PNG}
\end{center}
\caption{push to git hub}\label{exampleplot}
\end{figure}
\begin{figure}[!htbp]
\begin{center}
\includegraphics[scale=0.8]{test_add.PNG}
\end{center}
\caption{push to git hub result}\label{exampleplot}
\end{figure}


\section{Write a report about the utility git}

I use a lot of shift-delete when deleting a file. But there have been several problems with this. When I tried to find the file, I put the file on the desktop and finished working. After I finished changing the file location, I thought I saved it and completely erased it from my desktop. But there was nothing in the changed location. I tried to find it and even bought a paid program. But it doesn't work well. 
So I got interest in rewinding the file. As a result, two instructions came out. Reset and Revert.

\begin{enumerate}
\item Git reset $<$option$>$ $<$commit you want to go back$>$

Reset will reset the repository to commit, and history after that commit will be lost. Git reset also has three options.

\begin{itemize}
\item git reset --hard

It erases all the contents after the history to return. This is a command such as Shift-delete.
\end{itemize}
\begin{itemize}
\item git reset --soft

Unlike Hard, the content after the history is not erased, and the index of the content remains. It is still available for committing again.
\end{itemize}
\begin{itemize}
\item git reset --mixed

The contents after the history are not erased, but the index of the contents is initialized, so you have to add the changed contents again to commit.
\end{itemize}

\item Git revert $<$commit you want to go back$>$

Revert is a state in which the commit after the history has not been cleared yet. There is a new history on the history.
\end{enumerate}

\section{my git}
\begin{figure}[!htbp]
\begin{center}
\includegraphics[scale=0.4]{git.PNG}
\end{center}
\caption{screenshot of my project at github}\label{exampleplot}
\end{figure}

link: https://github.com/tjswns494/assignment01.git



\end{document}
